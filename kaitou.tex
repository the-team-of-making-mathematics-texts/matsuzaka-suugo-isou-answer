\documentclass[dvipdfmx,uplatex,11pt]{jsarticle}
%
\usepackage[dvipdfmx]{graphicx}
\usepackage{amsmath,amssymb,amsthm}
\usepackage{enumitem}
\usepackage{mathtools}
\usepackage{wrapfig}
\usepackage{bm}
\usepackage{ascmac}
\setcounter{tocdepth}{2}
\usepackage{physics}
\usepackage{geometry}
\usepackage{framed}
\usepackage{latexsym}
\everymath{\displaystyle}
%
\geometry{left=10mm,right=10mm,top=5mm,bottom=10mm}
\title{集合・位相入門:解答集}
\author{}
\begin{document}
\maketitle
\section{集合と写像}
\subsection{p11}
(1)
\begin{leftbar}
\begin{proof}
必要条件であることを示す.$a \in A$を前提に、$\{ a\} \subset A$の定義が成り立つことを示す.$\{ a \} \subset A$の定義は任意の$x$について$x \in \{a\} \Rightarrow x  \in A$. \par
今,$a \in A$を前提とするので,当然$x=a \Rightarrow x \in A$.$x \in  \{a\} \Rightarrow x=a$と合わせると$x \in \{a\} \Rightarrow x \in A$.これは任意の$x$について成り立つので上の定義が成り立つ.\par
十分条件であることを示す.$\{a\} \subset A$を前提に$a \in A$が成り立つことを示す.\par
今,$\{a\} \subset A$を前提とするので定義より任意の$x$について$x \in \{a\} \Rightarrow x \in A$.また,$a \in \{a\}$なので合わせると$a \in A$が成り立つ.
\end{proof}
\end{leftbar}
(2)
\begin{leftbar}
    \[
        \{x \mid x \in \mathbb{R}, (x-1)(x-2)(x-3)=0\}
    \]
\end{leftbar}
(3)
\begin{leftbar}
    (a)\quad $x^6 -1 =0$について,解を$x=r(\cos \theta + i \sin \theta)$($r>0$)とすると,
    \[
        x^6 = r^6 (\cos 6 \theta + i \sin 6 \theta)=1 =\cos 0 + i \sin 0
    \]
    と$r>0$より,$r=1$となる.ここで,
    \begin{gather*}
        6 \theta = 0 + 2k\pi \, (0 \le \theta <2 \pi ) \, (k \in \mathbb{Z}) \\
        \therefore ~ \theta =\frac{k}{3}\pi 
    \end{gather*}
    であり,これを満たす$k$は$k=0,1,2,3,4,5$である.ゆえに,
    \[
        \{x \mid x \in \mathbb{C},x^6=1\} = \left \{1,-1,\pm \frac{1}{2} \pm \frac{\sqrt{3}}{2} i \right \}
    \]
    (b) 
    \begin{align*}
    i(x+i)^4&=i(x^4 + 4 x i^3 +6 x^2 i^2 + 4x^3 i +i^4) \\
    & = i(x^4 -4 xi -6x^2 + 4x^3 i+1) \\
    & = (x^4-6x^2+1)i + 4x-4x^3
    \end{align*}
    ここで,$i(x+i)^4 \in \mathbb{R}$となるための必要十分条件は,$x^4-6x^2+1=0$となることで,この方程式を解くと,
    \begin{align*}
        x^2 &= \frac{6\pm \sqrt{32}}{-2} \\
        & = 3 \pm 2 \sqrt{2} \\
        \therefore & x= \sqrt{2}+1 ,-\sqrt{2}-1,1-\sqrt{2},\sqrt{2}-1
    \end{align*}
    これより,
    \[
        \{x \mid x \in \mathbb{R},i(x+i)^4 \in \mathbb{R}\} = \left \{  \sqrt{2}+1 ,-\sqrt{2}-1,1-\sqrt{2},\sqrt{2}-1 \right \}
        \]
(c) \quad $y^3 =2$を$y \in \mathbb{R}$のもとで解くと,$y=\sqrt[3]{2}$であり,$\sqrt[3]{2} \notin \mathbb{Q}$である.
もし$\sqrt[3]{2} \in \mathbb{Q}$であるとすると,
\[
    \exists a,b \in \mathbb{Z} ~ \mathrm{s.t} ~ \sqrt[3]{2} = \frac{b}{a} \iff a^3 + a^3=b^3
\]
であり,これは$\mathrm{Fermat}$の最終定理に矛盾する.
\[
    \therefore \, \{y \mid y \in \mathbb{Q},y^3=2 \} = \varnothing
\]
\end{leftbar}
\begin{leftbar}
    $z \in \mathbb{Z}$のとき,$f(z)=2^z$とおくと,$f$は増加函数である.これと,
    \[
        2^{-4} < 0.1 < 2^{-3} < 2^6<100<2^7
    \]
    により,
    \[
        \{z \mid z \in \mathbb{Z},0.1<2^z<100\} = \{-3,-2,-1,0,1,2,3,4,5,6\}
    \]
\end{leftbar}
\begin{leftbar}
    $k \in \mathbb{N}$として,$n=4k-3,4k-2,4k-1,4k$の場合を調べる.\par 
    $n =4k-3$のとき,
    \begin{align*}
        i^n & = (i)^{4k-3} \\
        & = i^{-3} \\
        &=\frac{1}{-i} = i
    \end{align*}
    である.$n =4k-2$のとき,
    \begin{align*}
        i^n & = (i)^{4k-2} \\
        & = i^{-2} \\
        &=-1
    \end{align*}
    である.$n =4k-1$のとき,
    \begin{align*}
        i^n & = (i)^{4k-1} \\
        & = i^{-1} \\
        &=\frac{1}{i}=-i
    \end{align*}
    である.$n =4k$のとき,
    \begin{align*}
        i^n & = (i)^{4k} \\
        & = 1 
    \end{align*}
    である.以上の議論により,
    \[
        \{n \mid n \in \mathbb{N},i^n =-1\} = \{2,6,10,14,18,\cdots,4n-2,\cdots \}
    \]
    と表される.
\end{leftbar}
(f)
\begin{leftbar}
    $n \in \mathbb{N}$のもとで,
    \begin{align*}
        i^{2n}&=(-1)^n \\
        & =
        \begin{cases}
            1 & (n\text{が偶数のとき})\\
            -1 & (n\text{が奇数のとき})
        \end{cases}
    \end{align*}
となるため,
\[
    \{n \mid n \in \mathbb{N},i^{2n}=i\} = \varnothing
\]
である.
\end{leftbar}
\newpage
問4:(i)
\begin{leftbar}
    \begin{proof}
    $x=x_1 + x_2 \sqrt{2}$,$y=y_1 + y_2 \sqrt{2}$~($x_1,x_2,y_1,y_2 \in \mathbb{Q}$)とおく.このとき,
    \[
        x+y = (x_1+y_1)+(x_2+y_2)\sqrt{2}
    \]
    となり,$x_1+y_1,x_2+y_2 \in \mathbb{Q}$なので,$x+y \in A$となる.また,
    \[
        x+y = (x_1-y_1)+(x_2-y_2)\sqrt{2}
    \]
    となり,$x_1-y_1,x_2-y_2 \in \mathbb{Q}$なので,$x-y \in A$となる.また,
    \[
        xy= (x_1 y_1 +2x_2 y_2) + (x_1y_2+x_2y_2)\sqrt{2}
    \]
    となり,$x_1 y_1 +2x_2 y_2,x_1y_2+x_2y_2\in \mathbb{Q}$なので,$xy \in A$となる.

    以上の議論により,(i)の主張がたしかめられた.
    \end{proof}
\end{leftbar}
(ii)
\begin{leftbar}
    \begin{proof}
        $x \in A$,$x \ne 0$であるから,$a,b \in \mathbb{Q} \setminus \{0\}$として,
        \[
            x=a+bi
        \]
        とかける.このとき,
        \[
        x^{-1}=\frac{1}{a+bi}=\frac{a-bi}{a^2+b^2} = \frac{a}{a^2+b^2} - \frac{b}{a^2+b^2} i
        \]
        であるから,$x^{-1} \in A$である.よって命題の主張が正しいことが示された.
    \end{proof}
\end{leftbar}
\subsection{p39}
\begin{screen}
    (1) \quad (4.2) \\
{\it Proof.}
\begin{align*}
    & b \in f(P_1 \cup P_2) \\
    \iff & \exists a \in P_1 \cup P_2 ~ \mathrm{s.t} ~ f(a)=b \\
    \iff & \exists a \in P_1 ~\mathrm{s.t.} ~ f(a)=b \lor \exists c \in P_2 ~ \mathrm{s.t} ~ f(c)=b \\
    \iff & b \in f(P_1) \cup f(P_2) 
\end{align*}
\dotfill 
\begin{align*}
\because  \quad  & b \in f(P_1) \lor b \in f(P_2) \\
\iff & \exists a ~ \mathrm{s.t.} ~ a \in P_1 \lor a \in P_2 ,b = f(a)
\end{align*}
\end{screen}
\begin{screen}
    (1) \quad (4.2)' \\
    {\it Proof.}
\begin{align*}
    & a \in f^{-1} (Q_1 \cup Q_2) \\
    \iff & f(a) \in Q_1 \cup Q_2 \\
    \iff & f(a)  \in Q_1 \lor f(a) \in Q_2 \\
    \iff & a \in f^{-1} (Q_1) \cup f^{-1} (Q_2)
\end{align*}
\end{screen}
\begin{screen}
    (2) \quad (4.5) \\
    {\it Proof.}
    \begin{align*}
        & a \in P \\
       & \Longrightarrow  f(a) \subset f(P) \\
       & \Longrightarrow a \in f^{-1} (f(P))
    \end{align*}
    \dotfill 
    \[
        \because \quad f^{-1} (f(P)) = \{ a \mid f(a)\in f(P)\}
        \]
    また,$f \colon \mathbb{R} \ni x \mapsto x^2 \in \mathbb{R}$は逆が成り立たない例である.\par
    $P=[1,2]$とすると,$f(P)=[1,4]$,$f^{-1}(f(P))=[-2,-1] \cup [1,2]$.
\end{screen}
\subsection{p151}
1.
\begin{leftbar}
    \begin{proof}
        $A$に関する条件により,$k \in \mathbb{N}$を用いて,
        \[
            A = \{a_1  , \dots , a_k\}
        \]
        とかける.

        ここで,
        \[
            A^c = \mathbb{R}^n \setminus A
        \]
        の点$a$を任意にとる.このとき,ある$\varepsilon >0$を
        \[
             \Bigl (0< \Bigl) ~ \varepsilon < \min \{ d(a,a_1),\dots,d(a,a_k)\}
        \]
        を満たすようにとると,
        \begin{align*}
            & B(a;\varepsilon) \cap A = \varnothing \\
            \iff & B (a;\varepsilon) \subset A^c
        \end{align*}
        となるため, $A^c$は$\mathbb{R}^n$の開集合である.よって,$A$は閉集合であることがただちに従う.
    \end{proof}
\end{leftbar}
\newpage
2.
\begin{leftbar}
    \begin{proof}
    $a,b$は$\mathbb{R}^n$の相異なる2点だから,$d(a,b)>0$となることはよい.

    ここで,$0< \varepsilon < \frac{d(a,b)}{2}$なる$\varepsilon$をとり,
    \[
        U \coloneqq B(a,\varepsilon),\quad V \coloneqq B (b,\varepsilon)
    \]
    とする.このとき,$U,V$は開集合である.
    
    また,$x \in U \cap V$なる$x \in \mathbb{R}^n$が存在すると仮定すると,
    \[
        d(x,a)<\varepsilon ,\quad d(x,b)<\varepsilon
    \]
    となる.このとき,三角不等式により,
    \[
        d(a,b)  \le d(x,a)+d(x,b) <\varepsilon + \varepsilon = 2 \varepsilon = d(a,b)
    \]
    となり,これは矛盾.ゆえに
    \[
        U \cap V = \varnothing 
    \]
    となり,ただちに主張が従う.
    \end{proof}
\end{leftbar}
\newpage
3
\begin{leftbar}
    \begin{proof}
        まず,前半の主張について示す.

        $x=(x_1  , \dots , x_n)\in (a_1 , b_1) \times \dots \times (a_n , b_n)$とする.このとき,
        \[
            \varepsilon \coloneqq \min \{ \abs{x_1-a},\dots,\abs{x_n -a},\abs{x_1-b},\dots, \abs{x_n-b} \}
        \]
        なる$\varepsilon$を任意にとる.
        このとき,
        \[
            B(x;\varepsilon) \subset (a_1 , b_1) \times \dots \times (a_n , b_n)
        \]
        であるから,$(a_1 , b_1) \times \dots \times (a_n , b_n)$は開集合である.

        後半の主張について示す.
        
        $ y = (y_1 , \dots + y_n ) \notin [a_1 , b_1] \times \dots \times [a_n , b_n]$とする.

        このとき,ある$j \in \{1,\dots n\}$について,$y_j \notin [a_j , b_j]$であることはよい.さらに,
        \[
            \delta \coloneqq \min  \{ \abs{y_j -a},\abs{y_j - b}\}
        \]
        とすると,$ B(y , \delta ) \subset {[a_1,b_1] \times \dots \times [a_n , b_n]}^c$である.
        したがって,$[a_1,b_1] \times \dots \times [a_n , b_n]$は閉集合である.
    \end{proof}
    \end{leftbar}
    \subsection{p206}
    1.
    \begin{leftbar}
        \begin{proof}
    ある$S$及び$\varnothing$以外の$S$のある部分集合$M$に対して,$M^f=\varnothing$と仮定すると,
    $\mathring{M}\subset M,M\subset\overline{M}$(すなわち,$\mathring{M}\subset\overline{M}$)かつ$M^f=\overline{M}\setminus\mathring{M}=\varnothing$なので,
$M=\mathring{M}=\overline{M}$である.ここで,$M$は$S$でも$\varnothing$でもない開かつ閉の$S$の部分集合となるので,$S$の連結性に矛盾する.
従って,$S$でも$\varnothing$でもない任意の$S$の部分集合$M$について,$M^f=\varnothing$となる.
        \end{proof}
    \end{leftbar}
\end{document}